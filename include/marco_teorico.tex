\chapter{Marco Teórico}

\section{Machine Learning}

El \textit{Machine Learning} o Aprendizaje automático el departamento de informática de la Universidad Técnica Federico Santa María (UTFSM) lo define como ''una subrama de la Inteligencia Artificial (IA), y como su nombre lo indica, esta tecnología trata de darle a la máquina la capacidad de aprender. El aprendizaje automático se basa algoritmos que aprenden y realizan predicciones. Tales algoritmos operan mediante la construcción de un modelo basados en conjuntos de datos de entrenamiento'' \cite{machinelearningUTFSM}.

En otras palabras es una ciencia que permite el estudio del comportamiento o patrones presentes en diversos tipos de datos, para poder automatizar diversos procesos. Además esto implica un aprendizaje continuo que va mejorando con cada iteración haciéndolo más inteligente y capaz de resolver diferentes problemas en base a lo que va aprendiendo.

Su utilización en diversos ámbitos tiene variadas ventajas, esta permite mejorar la gestión organizacional, facilitar la toma de diferentes decisiones, automatizar y acelerar procesos, entre muchas otras. Pero es como de esperarse también presenta desventajas, las cuales vienen muy de la mano con las decisiones humanas, ya que estas decisiones afectan la resolución que toma el algoritmo en las tareas que se le asignan. Una mala decisión humana puede influir en malos resultados del algoritmo y un mal desempeño en las tareas asignadas.

Este aprendizaje se divide en dos, el aprendizaje supervisado y el aprendizaje no supervisado.

\subsection{Aprendizaje Supervisado}

El aprendizaje supervisado consiste en intentar deducir a partir de datos de entrenamiento o ejemplo una función que clasifique datos sin una clasificación previa. En este caso los datos están compuestos por dos partes, por un lado están los diferentes atributos, ya sean numéricos o categóricos, y por otro lado una etiqueta que clasifica el dato. Entonces lo que se hace es determinar mediante los diferentes atributos el valor de la etiqueta, para luego poder predecir la clasificación de datos no categorizados. Un ejemplo de algoritmo de aprendizaje no supervisado es el de \textit{K} vecinos mas cercanos.

\subsection{Aprendizaje No Supervisado}

El aprendizaje no supervisado, a diferencia del supervisado no posee un conocimiento a priori de una clasificación de los datos, por lo tanto un modelo se ajusta al número de observaciones que contiene un data set. En este tipo de aprendizaje solo se cuenta con diferentes atributos, sobre los cuales se buscan semejanzas para poder clasificarlos y crear agrupaciones o clústers. Algunos ejemplos de algoritmos de aprendizaje no supervisado son K-Means y la extensión propuesta en \cite{Pelleg00x-means:extending}, X-Means.
\chapter{Propuesta de Solución}

Este trabajo presenta un estudio tanto de establecimientos educacionales como de sus alumnos, para poder determinar la estructura que estos presentan y poder generar un modelo de selección de colegios basado en los grupos que se generen a partir de los conjuntos de datos iniciales.

\section{Metodología de trabajo}

Con la finalidad de llevar a cabo los objetivos planteados para este estudio se optó por utilizar una metodología propia basada en la ya existe CRSIP-DM. 

El trabajo se desarrolla de la siguiente forma:

\begin{itemize}
    \item \textbf{Comprensión de datos:} esta etapa consiste básicamente en la colección de datos desde diferentes fuentes. En este caso fueron utilizadas las bases de datos de establecimientos y matrículas de la Región Metropolitana facilitadas por el CIAE. Además se complementó la base de datos de los colegios con la información pública disponible en la página Web del Ministerio de Educación de Chile (MIME \cite{MIME}), los cuales fueron obtenidos mediante la técnica de \textit{Web scraping}.
    
    \item \textbf{Preparación de datos:} en esta fase, a partir de los datos obtenidos anteriormente, se seleccionaron los atributos más relevantes para el estudio y se realizó un proceso de limpieza y estandarización de los datos. También se agregaron algunos atributos que fueron derivados o calculados a partir de los datos extraídos originalmente, como los atributos s\_basica, s\_media, completa, IDE\_rango y dist\_percentil\_75 en el caso de los establecimientos, y la sobre\_edad en el de las matrículas. 
    
    Además, los atributos fueron categorizados según el porcentaje de repetición que tienen cada uno de sus valores. Es decir, se contabilizó la cantidad de repeticiones de un valor para cada atributo y se calculó su porcentaje. Esto se hizo para ver de que manera influyen las variables que tienen un solo valor que predomina por sobre el resto al momento de ejecutar los algoritmos de agrupamiento. Las categorías según su importancia son:
    
    \begin{itemize}
    \item Baja: cuando el porcentaje de aparición de un valor es mayor o igual a 95\%. 
    \item Media: cuando el porcentaje de aparición de un valor es mayor o igual a 85\% y menor que 95\%.
    \item Alta: cuando ninguno de los valores de una variable alcanza un porcentaje de aparición mayor o igual a 85\%.
\end{itemize}

Las tablas \ref{tab:atributos_establecimientos} y \ref{tab:atributos_matriculas} presentan los atributos correspondientes a los establecimientos y matrículas, respectivamente.

\begin{footnotesize}
\begin{longtable}{|p{0.21\textwidth}|p{0.3\textwidth}|p{0.15\textwidth}|p{0.15\textwidth}|}
\caption{Atributos que componen la base de datos de los establecimientos.}\label{tab:atributos_establecimientos}\\
\hline
\endfirsthead
\caption[]{Atributos que componen la base de datos de los establecimientos. (continuación)}\\
\hline
\endhead
\hline
\multicolumn{4}{|c|}{continúa $\ldots$}\\
\hline
\endfoot
\hline
\endlastfoot
\textbf{Atributo}  & \textbf{Descripción} & \textbf{Importancia} & \textbf{Fuente}\\ \hline
area\_metropolitana\_rbd & Pertenencia al área metropolitana. &Media & CIAE \\ \hline
cod\_depe & Código de dependencia del establecimiento. & Alta & CIAE \\ \hline
gen\_rbd & Género del establecimiento. & Baja & CIAE \\ \hline
mat\_total & Matrícula total de alumnos. & Alta & MIME \\ \hline
prom\_alu\_cur & Promedio de alumnos por curso. & Alta & MIME \\ \hline
pago\_mat & Nivel de pago de matrícula. & Alta & MIME \\ \hline
pago\_men & Nivel de pago de mensualidad. & Alta & MIME \\ \hline
becas\_disp & Becas disponibles en el establecimiento. & Alta & MIME \\ \hline
convenio\_sep & Posee convenio de subvención escolar preferencial (SEP). & Alta & MIME \\ \hline
deportivo & Nivel deportivo del establecimiento. & Alta & MIME \\ \hline
req\_papeles & Requisitos de papeles para postular. & Media & MIME \\ \hline
req\_pruebas & Requisitos de prueba para postular. & Baja & MIME \\ \hline
req\_entrevista & Requisitos de entrevista para postular. & Alta & MIME \\ \hline
req\_pago & Requisitos de pago para postular. & Media & MIME \\ \hline
req\_otros & Requisitos de cualquier tipo que no clasifique en las categorías anteriores. & Baja & MIME \\ \hline
enf\_académico & Enfoque académico. & Media & MIME \\ \hline
enf\_valorico & Enfoque valórico. & Alto & MIME \\ \hline
enf\_laboral & Enfoque laboral. & Baja & MIME \\ \hline
enf\_otros & Enfoque de otro tipo que no clasifique en las categorías anteriores. & Baja & MIME \\ \hline
apoyo\_tutorias & Ofrece ayuda a los alumnos mediante tutorías. & Media & MIME \\ \hline
apoyo\_especialistas & Ofrece ayuda a los alumnos mediante especialistas. & Media & MIME \\ \hline
apoyo\_otros & Ofrece ayuda a los alumnos de cualquier otra forma que no clasifique en la categorías anteriores. & Baja & MIME \\ \hline
s\_basica & Establecimiento de enseñanza básica. & Alta & Derivados de datos CIAE. \\ \hline
s\_media & Establecimiento de enseñanza media. & Alta & Derivados de datos CIAE. \\ \hline
completa & Establecimiento de enseñanza completa. & Alta & Derivados de datos CIAE. \\ \hline
\end{longtable} 
\end{footnotesize}

\begin{footnotesize}
\begin{longtable}{|p{0.2\textwidth}|p{0.3\textwidth}|p{0.15\textwidth}|p{0.15\textwidth}|}
\caption{Atributos que componen la base de datos de las matrículas.}\label{tab:atributos_matriculas}\\
\hline
\endfirsthead
\caption[]{Atributos que componen la base de datos de las matrículas. (continuación)}\\
\hline
\endhead
\hline
\multicolumn{4}{|c|}{continúa $\ldots$}\\
\hline
\endfoot
\hline
\endlastfoot
\textbf{Atributo}  & \textbf{Descripción} & \textbf{Importancia} & \textbf{Fuente} \\ \hline
area\_metropolitana\_alu & Pertenencia al área metropolitana. & Alta & CIAE \\ \hline
gen\_alu & Género del establecimiento. & Alta & CIAE \\ \hline
cod\_sec & Código del sector económico. & Baja & CIAE \\ \hline
cod\_espe & Código de especialidad. & Baja & CIAE \\ \hline
cod\_rama & Código de rama. & Baja & CIAE \\ \hline
grado\_sep & Corresponde a un nivel SEP. & Alta & CIAE \\ \hline
beneficiario\_sep & Indicador del alumno beneficiario de la SEP. & Alta & CIAE \\ \hline
criterio\_sep & Criterio por el cual se considera prioritario. & Alta & CIAE \\ \hline
\end{longtable} 
\end{footnotesize}

Además se añadieron variables que relacionan a ambas entidades, las variables de relación establecimiento-matrícula. En el caso de los establecimientos se agregó la sobre edad promedio, la distancia máxima a la que viven el 75\% de los alumnos de cada colegio y el índice de desarrollo de la educación (IDE) por rangos. Y por otro lado, para las matrículas se añadió la distancia a la que vive el alumno del colegio, el nivel de sobre edad\footnote{Diferencia entre la edad actual y la edad esperada para el curso en el que se encuentra.} y el nivel de copago (matrícula y mensualidad). Estos se pueden ver en las tablas \ref{tab:atributos_relacion_establecimientos} y \ref{tab:atributos_relacion_matriculas}, respectivamente.

\begin{footnotesize}
\begin{longtable}{|p{0.21\textwidth}|p{0.3\textwidth}|p{0.15\textwidth}|p{0.15\textwidth}|}
\caption{Atributos de relación establecimiento-matrícula para la base de datos de los establecimientos.}\label{tab:atributos_relacion_establecimientos}\\
\hline
\endfirsthead
\caption[]{Atributos de relación establecimiento-matrícula para la base de datos de los establecimientos. (continuación)}\\
\hline
\endhead
\hline
\multicolumn{3}{|c|}{continúa $\ldots$}\\
\hline
\endfoot
\hline
\endlastfoot
\textbf{Atributo}  & \textbf{Descripción} & \textbf{Fuente}\\ \hline
sobre\_edad\_prom & Sobre edad promedio de los alumnos que estudian en el establecimiento. & Calculado con datos CIAE. \\ \hline
IDE\_rango & Índice de Desarrollo de la Educación para Todos por rango. & Calculado con datos CIAE. \\ \hline
dist\_percentil\_75 & Distancia del percentil 75 de los alumnos que asisten al establecimiento. & Calculado con datos CIAE.\\ \hline
\end{longtable} 
\end{footnotesize}

\begin{footnotesize}
\begin{longtable}{|p{0.2\textwidth}|p{0.3\textwidth}|p{0.15\textwidth}|p{0.15\textwidth}|}
\caption{Atributos de relación establecimiento-matrícula para la base de datos de las matrículas.}\label{tab:atributos_relacion_matriculas}\\
\hline
\endfirsthead
\caption[]{Atributos de relación establecimiento-matrícula para la base de datos de las matrículas. (continuación)}\\
\hline
\endhead
\hline
\multicolumn{3}{|c|}{continúa $\ldots$}\\
\hline
\endfoot
\hline
\endlastfoot
\textbf{Atributo}  & \textbf{Descripción} & \textbf{Fuente} \\ \hline
sobre\_edad & Diferencia entra la edad actual y la esperada para el nivel. & Calculado con datos CIAE.\\ \hline
dist\_actual & Distancia del alumno a su establecimiento actual. & CIAE \\ \hline
pago\_mat & Nivel de pago de matrícula. & MIME \\ \hline
pago\_men & Nivel de pago de mensualidad. & MIME \\ \hline
\end{longtable} 
\end{footnotesize}

Adicionalmente se imputaron los campos de datos faltantes mediante el algoritmo MICE (\textit{Multiple Imputation by Chained Equations}) para eliminar los valores nulos.

\item \textbf{Modelado:} en esta etapa se escoge el algoritmo de agrupamiento X-Means, debido a que uno de los objetivos del estudio es poder encontrar sin una idea previa la estructura que poseen los establecimientos y los alumnos, el cual es ejecutado sobre el software RapidMiner. En el programa se transforman las variables categóricas a numéricas y se realiza una normalización. Además se ajustan los diferentes parámetros del algoritmo, incluyendo el número de variables que se utiliza (según el nivel de importancia de la variable).

\item \textbf{Análisis y resultados:} se realiza un análisis de los resultados obtenidos de la fase de modelado, los cuales deben ser capaces de satisfacer los objetivos planteados para el estudio.

\item \textbf{Conclusiones:} finalmente se deben realizar las conclusiones respectivas del trabajo efectuado y validar el cumplimiento de los objetivos que fueron planteados para el trabajo. 

\end{itemize}
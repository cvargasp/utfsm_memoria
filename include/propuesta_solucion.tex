\chapter{Propuesta de Solución}

Este trabajo presenta un estudio tanto de establecimientos educacionales como de sus alumnos, para poder determinar la estructura que estos presentan y poder generar un modelo de selección de colegios basado en los grupos que se generen a partir de los conjuntos de datos iniciales.

\section{Metodología de trabajo}

Con la finalidad de llevar a cabo los objetivos planteados para este estudio se optó por utilizar una metodología propia basada en la ya existe CRSIP-DM. 

El trabajo se desarrolla de la siguiente forma:

\begin{itemize}
    \item \textbf{Comprensión de datos:} esta etapa consiste básicamente en la colección de datos desde diferentes fuentes. En este caso fueron utilizadas las bases de datos de establecimientos y matrículas de la Región Metropolitana facilitadas por el CIAE. Además se complementó la base de datos de los colegios con la información pública disponible en la página Web del Ministerio de Educación de Chile (MIME \cite{MIME}), los cuales fueron obtenidos mediante la técnica de \textit{Web scraping}.
    
    \item \textbf{Preparación de datos:} en esta fase, a partir de los datos obtenidos anteriormente, se seleccionaron los atributos más relevantes para el estudio y se realizó un proceso de limpieza y estandarización de los datos. Además, se agregaron algunos atributos derivados o calculados de los datos extraídos originalmente. 
    
    Para cada atributo se calculó su nivel de estabilidad, la cual mide cuán estable o constante es el atributo. La estabilidad se calcula mediante la división del número de filas no nulas del valor más frecuente y el número total de filas de datos no nulos. 
    
    Lo anteriormente descrito se resume en las tablas \ref{tab:atributos_establecimientos} y \ref{tab:atributos_matriculas}, las cuales presentan los atributos correspondientes a los establecimientos y matrículas, respectivamente, ordenados de menor a mayor según su estabilidad.

\begin{footnotesize}
\begin{longtable}{|p{0.21\textwidth}|p{0.3\textwidth}|p{0.15\textwidth}|p{0.15\textwidth}|}
\caption{Atributos que componen la base de datos de los establecimientos.}\label{tab:atributos_establecimientos}\\
\hline
\endfirsthead
\caption[]{Atributos que componen la base de datos de los establecimientos. (continuación)}\\
\hline
\endhead
\hline
\multicolumn{4}{|c|}{continúa $\ldots$}\\
\hline
\endfoot
\hline
\endlastfoot
\textbf{Atributo}  & \textbf{Descripción} & \textbf{Estabilidad} & \textbf{Fuente}\\ \hline
mat\_total & Matrícula total de alumnos. & 0,44\% & MIME \\ \hline
prom\_alu\_cur & Promedio de alumnos por curso. & 5,22\% & MIME \\ \hline
nivel\_deportivo & Nivel deportivo del establecimiento. & 48,36\% & MIME \\ \hline
nivel\_ensenanza & Nivel de enseñanza que ofrece el establecimiento. & 48,4\% & Derivado de datos CIAE. \\ \hline
cod\_depe & Código de dependencia del establecimiento. & 53,82\% & CIAE \\ \hline
enf\_valorico & Enfoque valórico. & 54,79\% & MIME \\ \hline
pago\_men & Nivel de pago de mensualidad. & 60,64\% & MIME \\ \hline
req\_entrevista & Requisitos de entrevista para postular. & 62,81\% & MIME \\ \hline
becas\_disp & Becas disponibles en el establecimiento. & 64,07\% & MIME \\ \hline
convenio\_sep & Posee convenio de subvención escolar preferencial (SEP). & 71,91\% & MIME \\ \hline
pago\_mat & Nivel de pago de matrícula. & 77,66\% & MIME \\ \hline
enf\_academico & Enfoque académico. & 84,62\% & MIME \\ \hline
area\_metropolitana\_rbd & Pertenencia al área metropolitana. & 85,74\% & CIAE \\ \hline
apoyo\_tutorias & Ofrece ayuda a los alumnos mediante tutorías. & 86,65\% & MIME \\ \hline
req\_papeles & Requisitos de papeles para postular. & 87,57\% & MIME \\ \hline
req\_pago & Requisitos de pago para postular. & 89,12\% & MIME \\ \hline
apoyo\_especialistas & Ofrece ayuda a los alumnos mediante especialistas. & 90,18\% & MIME \\ \hline
gen\_rbd & Género del establecimiento. & 94,54\% & CIAE \\ \hline
req\_otros & Requisitos de cualquier tipo que no clasifique en las otras categorías. & 95,02\% & MIME \\ \hline
enf\_laboral & Enfoque laboral. & 95,21\% & MIME \\ \hline
req\_prueba & Requisitos de prueba para postular. & 95,94\% & MIME \\ \hline
enf\_otros & Enfoque de otro tipo que no clasifique en las otras categorías. & 96,32\% & MIME \\ \hline
apoyo\_otros &  Ofrece ayuda a los alumnos de cualquier otra forma que no clasifique en las otras categorías. & 98,79\% & MIME \\ \hline
\end{longtable} 
\end{footnotesize}

\begin{footnotesize}
\begin{longtable}{|p{0.2\textwidth}|p{0.3\textwidth}|p{0.15\textwidth}|p{0.15\textwidth}|}
\caption{Atributos que componen la base de datos de las matrículas.}\label{tab:atributos_matriculas}\\
\hline
\endfirsthead
\caption[]{Atributos que componen la base de datos de las matrículas. (continuación)}\\
\hline
\endhead
\hline
\multicolumn{4}{|c|}{continúa $\ldots$}\\
\hline
\endfoot
\hline
\endlastfoot
\textbf{Atributo}  & \textbf{Descripción} & \textbf{Estabilidad} & \textbf{Fuente} \\ \hline
IPAE & Indicador de barrio crítico. & 0,6\% & CIAE \\ \hline
edad\_alu & Edad del alumno. & 8,79 & CIAE \\ \hline
gen\_alu & Género del alumno. & 9,04\% & CIAE \\ \hline
criterio\_sep & Criterio por el cual se considera prioritario. & 9,23\% & CIAE \\ \hline
beneficiario\_sep & Indicador del alumno beneficiario de la SEP. & 50,73\% & CIAE \\ \hline
gse\_decil\_nac & Grupo Socio Económico del alumno. & 64,04\% & CIAE \\ \hline
area\_metropolitana\_alu & Pertenencia al área metropolitana. & 90,48\% & CIAE \\ \hline
cod\_sec & Código del sector económico. & 94,81\% & CIAE \\ \hline
cod\_espe & Código de especialidad. & 94,81\% & CIAE \\ \hline
cod\_rama & Código de rama. & 94,81\% & CIAE \\ \hline
grado\_sep & Corresponde a un nivel SEP. & 100\% & CIAE \\ \hline
\end{longtable} 
\end{footnotesize}

Además de los atributos propios de cada establecimineto y matrículas se añadieron variables que relacionan a ambas entidades, las variables de relación establecimiento-matrícula. En el caso de los establecimientos se agregó la sobre edad promedio, el GSE que predomina en el establecimiento, la desviación de la moda GSE, los porcentajes de hombres y mujeres, la distancia máxima a la que viven el 75\% de los alumnos de cada colegio y el índice de desarrollo de la educación (IDE) por rangos. Por otro lado, para las matrículas se añadió la distancia a la que vive el alumno del colegio, el nivel de sobre edad\footnote{Diferencia entre la edad actual y la edad esperada para el curso en el que se encuentra.}, la dependencia del colegio y el nivel de copago (matrícula y mensualidad). Estos se pueden ver en las tablas \ref{tab:atributos_relacion_establecimientos} y \ref{tab:atributos_relacion_matriculas}, respectivamente.

\begin{footnotesize}
\begin{longtable}{|p{0.16\textwidth}|p{0.35\textwidth}|p{0.15\textwidth}|p{0.15\textwidth}|}
\caption{Atributos de relación establecimiento-matrícula para la base de datos de los establecimientos.}\label{tab:atributos_relacion_establecimientos}\\
\hline
\endfirsthead
\caption[]{Atributos de relación establecimiento-matrícula para la base de datos de los establecimientos. (continuación)}\\
\hline
\endhead
\hline
\multicolumn{3}{|c|}{continúa $\ldots$}\\
\hline
\endfoot
\hline
\endlastfoot
\textbf{Atributo}  & \textbf{Descripción} & \textbf{Fuente}\\ \hline
sobre\_edad\_prom & Sobre edad promedio de los alumnos que estudian en el establecimiento. & Calculado con datos CIAE. \\ \hline
desviacion\_moda & Desviación que presenta el grupo GSE con respecto a la moda. & Calculado con datos CIAE. \\ \hline
porc\_femenino & Porcentaje de mujeres del establecimiento. & Calculado con datos CIAE. \\ \hline
porc\_masculino & Porcentaje de hombres del establecimiento. & Calculado con datos CIAE. \\ \hline
dist\_percentil\_75 & Distancia del percentil 75 de los alumnos que asisten al establecimiento. & Calculado con datos CIAE.\\ \hline
IDE\_rango & Índice de Desarrollo de la Educación para Todos por rango. & Calculado con datos CIAE. \\ \hline
gse\_moda\_nac & Grupo Socio Económico que más se repite en el establecimiento. & Calculado con datos CIAE \\ \hline
\end{longtable} 
\end{footnotesize}

\begin{footnotesize}
\begin{longtable}{|p{0.2\textwidth}|p{0.3\textwidth}|p{0.15\textwidth}|p{0.15\textwidth}|}
\caption{Atributos de relación establecimiento-matrícula para la base de datos de las matrículas.}\label{tab:atributos_relacion_matriculas}\\
\hline
\endfirsthead
\caption[]{Atributos de relación establecimiento-matrícula para la base de datos de las matrículas. (continuación)}\\
\hline
\endhead
\hline
\multicolumn{3}{|c|}{continúa $\ldots$}\\
\hline
\endfoot
\hline
\endlastfoot
\textbf{Atributo}  & \textbf{Descripción} & \textbf{Fuente} \\ \hline
dist\_actual & Distancia del alumno a su establecimiento actual. & CIAE \\ \hline
pago\_men & Nivel de pago de mensualidad. & MIME \\ \hline
cod\_depe & Dependencia del colegio al que asiste. & CIAE \\ \hline
sobre\_edad & Diferencia entra la edad actual y la esperada para el nivel. & Calculado con datos CIAE.\\ \hline
pago\_mat & Nivel de pago de matrícula. & MIME \\ \hline
\end{longtable} 
\end{footnotesize}

Adicionalmente se imputaron los campos de datos faltantes mediante el algoritmo MICE (\textit{Multiple Imputation by Chained Equations}) para eliminar los valores nulos.

\item \textbf{Modelado:} en esta etapa se escoge el algoritmo de agrupamiento X-Means, debido a que uno de los objetivos del estudio es poder encontrar sin una idea previa la estructura que poseen los establecimientos y los alumnos, el cual es ejecutado sobre el software RapidMiner. En el programa se transforman las variables categóricas a numéricas y se realiza una normalización. Además se ajustan los diferentes parámetros del algoritmo, incluyendo el número de variables que se utiliza (según la estabilidad de las variables propias).

\item \textbf{Análisis y resultados:} se realiza un análisis de los resultados obtenidos de la fase de modelado, los cuales deben ser capaces de satisfacer los objetivos planteados para el estudio.

\item \textbf{Conclusiones:} finalmente se deben realizar las conclusiones respectivas del trabajo efectuado y validar el cumplimiento de los objetivos que fueron planteados para el trabajo. 

\end{itemize}
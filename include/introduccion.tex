\chapter*{Introducci\'on}
\addcontentsline{toc}{chapter}{Introducci\'on}

En Chile los establecimientos educacionales se encuentran categorizados según su dependencia administrativa en municipal, particular subvencionado, particular pagado y corporación de administración delegada, los cuales en la Región Metropolitana se distribuyen de la siguiente manera 23,8\%, 65,2\%, 10\% y 1\% respectivamente \cite{estadisticasEducacion}. Por otro lado, los estudiantes matriculados en dichos colegios no poseen una categorización clara, y las clasificaciones más cercanas son por el tipo de colegio al cual asisten o por su nivel socioeconómico.

Debido a esto es que con este estudio se busca encontrar, a partir de diversas variables, la estructura que tienen los establecimientos en Chile y sus estudiantes, de manera de poder realizar una mejor clasificación. Para esto se tomaron diferentes fuentes de información, las cuales fueron corregidas y estandarizadas para poder trabajar de manera sencilla con ellas.

Primero se escogieron las variables de interés pertenecientes a cada establecimiento o estudiante (sin considerar las variables que los relacionan), para luego ver cuales eran realmente relevantes seleccionar para realizar, todo esto mediante un algoritmo de aprendizaje no supervisado con el fin de no tener una categorización previa de los datos y poder deducir una clasificación directamente de los datos.

Una vez obtenidos los resultados de la prueba anterior, se le agregan las variables de relación establecimiento-matrículas y se comparan, para analizar si al añadir este tipo de información genera un enriquecimiento de la categorización obtenida anteriormente.

Finalmente, con las geolocalizaciones de establecimientos y estudiantes se generan mapas superpuestos al mapa GSE de la Región Metropolitana para determinar la relación existente entre los clústers generados y el sector socioeconómico en el cual están situados los colegios y en los lugares que residen los estudiantes.

En el primer capítulo se realiza un acercamiento al problema y los objetivos del estudio. Luego, en el segundo, se expone el estado del arte sobre el problema de elección de colegios y se presenta un breve marco teórico conceptual sobre los diferentes tipos de aprendizaje que existen. El tercer capítulo se enfoca en la propuesta de solución planteada para el problema en estudio, destacando el porque y como se utilizó el algoritmo escogido. Finalmente, en el cuarto capítulo, se presentan los resultados obtenidos y los análisis correspondientes, para luego presentar las conclusiones del estudio.
\chapter*{Introducci\'on}
\addcontentsline{toc}{chapter}{Introducci\'on}

En Chile los establecimientos educacionales se encuentran categorizados según su dependencia administrativa en municipal, particular subvencionado, particular pagado y corporación de administración delegada, los cuales en la Región Metropolitana se distribuyen de la siguiente manera 23,8\%, 65,2\%, 10\% y 1\% respectivamente \cite{estadisticasEducacion}. Por otro lado, los estudiantes matriculados en dichos colegios no poseen una categorización clara, y las clasificaciones más cercanas son por el tipo de colegio al cual asisten o por su nivel socioeconómico. La falta de perfiles de establecimientos y matrículas, basados en múltiples atributos, dificulta la tarea de llevar a cabo políticas públicas enfocadas a ciertos perfiles educaciones y estudiantiles.

Debido a esto es que con este estudio se busca encontrar, a partir de diversas variables, la estructura que tienen los establecimientos en la Región Metropolitana y los estudiantes que a ellos asisten, de manera de poder realizar una mejor clasificación. 

Para esto se utilizó una metodología propia basada en CRISP-DM, en donde primero se tomaron diferentes fuentes de información, las cuales fueron corregidas y estandarizadas para poder trabajar de manera sencilla con ellas. A continuación de esto se escogieron variables de interés propias de cada establecimiento o estudiante, para poder estudiar su relevancia y también variables que las relacionan.

Luego, con el fin de no tener una categorización previa de los datos, se utilizó un algoritmo de aprendizaje no supervisado para poder deducir una clasificación directamente de ellos. Primero se realizó solo con las variables propias para así poder seleccionar las más relevantes. Después, el resultado obtenido se compara con los resultados que se obtienen al incluir las variables de relación, tanto en el caso de los establecimientos como de las matrículas, para analizar si al añadir este tipo de información genera un enriquecimiento de la categorización obtenida anteriormente.

Finalmente, con las geolocalizaciones de establecimientos y estudiantes se generan mapas superpuestos al mapa GSE de la Región Metropolitana para determinar la relación existente entre los clústers generados y el sector socioeconómico en el cual están situados los colegios y en los lugares que residen los estudiantes.

En el primer capítulo se realiza un acercamiento al problema y los objetivos del estudio. Luego, en el segundo, se expone el estado del arte sobre el problema de elección de colegios y se presenta un breve marco teórico conceptual sobre el aprendizaje automático, el clustering y la metodología CRISP-DM. El tercer capítulo se enfoca en la propuesta de solución planteada para el problema en estudio, describiendo la metodología utilizada para el desarrollo del problema. Finalmente, en el cuarto capítulo, se presentan los resultados obtenidos y los análisis correspondientes, para luego presentar las conclusiones del estudio.
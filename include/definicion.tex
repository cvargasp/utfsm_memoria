\chapter{Definici\'on del Problema}

\section{Contextualización}

\subsection{Datos a analizar}

Las bases de datos de establecimientos y matrículas fueron facilitadas por el Centro de Investigación Avanzada en Educación (CIAE). Donde la primera fue complementada con la información disponible en la página web del Ministerio de Educación de Chile, MIME, mediante la técnica de \textit{web scraping}.

La base de datos de establecimientos corresponde a los colegios de la región metropolitana que imparten enseñanza básica y media para niños y jóvenes, tanto humanísta científico como técnico profesional. Por otro lado las matrículas corresponden a estudiantes que se encuentran inscritos en los colegios anteriormente descritos. 

\section{Identificación del Problema}

\section{Objetivos Generales}

El objetivo principal es establecer las relaciones existentes entre perfiles de alumnos y grupos de establecimientos para que a partir de esto se puedan generar políticas públicas de acuerdo a la realidad escolar que se vive en Chile.

Para esto se utilizará un algoritmo de clusterización sobre establecimientos y matrículas, los cuales se asociarán entre sí para determinar los atributos que los relacionan.

\subsection{Objetivos Específicos}
\begin{enumerate}
\item Generar clústers de establecimientos mediante el uso del algoritmo X-Means y comparar los resultados según el número de atributos escogidos para cada prueba.
\item Generar clústers de matrículas mediante el uso del algoritmo X-Means y comparar los resultados según el número de atributos escogidos para cada prueba.
\end{enumerate}
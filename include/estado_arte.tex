\chapter{Estado del Arte}

En el transcurso de los años han habido varios investigadores interesados en conocer los diferentes factores que influyen en los padres al momento de elegir un colegio en Chile.

En el 2002 Sapelli y Torche \cite{SAPELLI2002} estudiaron los diferentes determinantes que inciden en la elección del tipo de colegio que realizan los padres al momento de matricular a sus hijos en un determinado establecimiento educacional. En este suponen que una mayor educación de los hijos proveerá a los padres  una probabilidad mayor de apoyo cuando estén en la vejez, por lo cual utilizan un modelo en donde se postula una función de utilidad para los padres. Dicha función depende del capital humano inicial de los hijos, el cual puede incrementarse con la educación, y del nivel de consumo presente. Para esto utilizaron diferentes fuentes de información, siendo las más relevantes la encuesta de caracterización socioeconómica (CASEN) de 1996 y los resultados del SIMCE, en donde solo consideraron los datos referentes a la elección de establecimientos de enseñanza básica (niños entre 7 y 14 años), en donde el nivel de cobertura es cercano al 100\%\footnote{98,2\% de cobertura educacional para el nivel de enseñanza básica en el año 1996. Fuente: CASEN 1996.} y se descarta la opción de no elegir un colegio. Los resultados que obtuvieron apuntan a que algunos de los factores más determinantes son el nivel de ingreso, la educación de los padres, la recepción de subsidios y la calidad del colegio. Además destacan que por ser los subsidios por colegios y no por alumno, es decir no son portables, genera que sea más difícil para las familias de menores recursos acceder a estos si deciden optar por un colegio donde el nivel de subsidio es menor. Otro punto importante que destacan es la alta sensibilidad que los padres demuestran respecto a la calidad de los colegios, aún sin conocer los resultados SIMCE, actúan de tal forma que hace pensar que los conocieran.

En el año 2009 Gallego y Hernando \cite{gallego2010school} buscando resolver la interrogante de cómo los padres escogen el colegio para sus hijos usaron un modelo basado en el desarrollado por McFadden \cite{McFadden74}, junto con las especificaciones planteadas por Berry, Levinsohn y Pakes \cite{berry1995automobile}. Para el estudio se consideraron diferentes variables, las cuales se pueden agrupar en dos grandes categorías: características del alumno y características del colegio. En ambos casos los datos son obtenidos del SIMCE del 2012 o calculados por los autores a partir de dichos datos para un universo de 70.000 alumnos de cuarto básico que asisten a 1.200 colegios. A partir del modelo y los datos utilizados se obtuvo que existen dos variables que afectan más al momento de escoger un colegio, las cuales son el resultado del establecimiento en las pruebas y la distancia entre el hogar y el colegio, en donde la primera variable se repite respecto al estudio \cite{SAPELLI2002}.

Dos años después, en el 2011, Daniel Gómez en conjunto con R. Chumacero y R. Paredes \cite{Chumacero20111103} realizan un estudio similar a los ya presentados, en donde consideran diversos factores que consideran los padres al escoger un determinado colegio. Dichos factores se pueden clasificar en características particulares de cada niño, las propias de cada establecimiento y las que asocian al niño con la escuela, como la distancia entre el hogar y el colegio. Para llevar a cabo esto establecieron una función similar a la presentada en \cite{SAPELLI2002}, donde se mide la utilidad de que un niño asista a un determinado colegio y que depende de los tres grupos de factores mencionados. Al igual que en trabajos anteriores fueron considerados datos de la encuesta CASEN y del SIMCE, ambos correspondientes al año 2003. Mediantes los estudios realizados llegaron a la conclusión de que de los factores analizados la localización, el precio, la calidad y la potencial competencia de los establecimientos son determinantes al momento de realizar la elección, pero los más valorados por los padres son la calidad y la distancia.

Al año siguiente Gómez, Chumacero y Paredes \cite{GOMEZ2012}, buscan determinar si el conocimiento de resultados de pruebas específicas (SIMCE) determina de manera importante la selección que realizan los padres sobre el colegio donde matricular a sus hijos. Para esto realizaron un estudio comparativo, tomando como base el estudio anterior y comparándolo con datos de 1996 (primer año donde se hicieron públicos los resultados del SIMCE, por lo cual no influyen en la elección de colegios de ese año). Del estudio se obtuvo que aún sin conocer los resultados los padres actúan como si los conocieran escogiendo escuelas de mayor calidad, tal como se obtuvo en \cite{SAPELLI2002}. Además, cuando los resultados de las pruebas se hicieron públicos, este pasó a ser un factor aún más determinante al momento de tomar una decisión.

Finalmente, uno de los trabajos más recientes en torno a la selección de colegios fue realizado por Canales, Bellei y Orellana \cite{canalesque}, donde a diferencia de los trabajos anteriormente señalados, este se enfoca en un sector social específico para determinar y comprender el sentido que tiene para los padres de clase media el elegir un colegio privado. Para este estudio utilizaron dos técnicas complementarias: grupo de discusión y entrevista focalizada, donde la primera apunta a conocer cuál es el valor o significado colectivo de la decisión y la segunda permite conocer como el sujeto entiende la decisión que esta tomando. Los resultados obtenidos son de un carácter preocupante, ya que la selección de colegios esta guiada por el interés del sector medio de distanciarse y diferenciarse de los más pobres, siendo esto una decisión netamente clasista. Además esta preocupa del lado de la educación, debido a que al parecer ni familias ni escuelas parecen orientadas a mejorar el nivel de educación.
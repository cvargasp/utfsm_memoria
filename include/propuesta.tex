\chapter{Propuesta}

\section{Pre-procesamiento de los datos}

Antes de ejecutar el algoritmo de clusterización se realiza un proceso ETL (extract, transform and load), el cual permite limpiar y estandarizar las bases de datos.

Se seleccionan atributos numéricos y categóricos para los establecimientos y se generan otros a partir de la información extraída del MIME\cite{MIME}, debido a que mucha de esta no se encuentra estandarizada. De la misma forma se seleccionan atributos numéricos y categóricos para las matrículas, y a partir de variables no seleccionadas se calculan nuevas como es el caso de la sobre edad. Además se incorpora de manera relacional con el establecimiento al que asisten su nivel de copago y la distancia a la cual viven del colegio. De igual manera, para los establecimientos se agregan atributos relacionales, como lo es la distancia a la que viven sus estudiantes en su percentil 75 y el índice de desarrollo de la educación (IDE) por rango. El detalle de los atributos seleccionados pueden ser encontrados en los anexos \ref{tab:atributos_establecimientos} y \ref{tab:atributos_matriculas} respectivamente.

\begin{footnotesize}
\begin{longtable}{|p{0.2\textwidth}|p{0.3\textwidth}|p{0.15\textwidth}|p{0.15\textwidth}|}
\caption{Atributos seleccionados y generados para la base de datos de establecimientos.}\label{tab:atributos_establecimientos}\\
\hline
\endfirsthead
\caption[]{Atributos seleccionados y generados para la base de datos de establecimientos. (continuación)}\\
\hline
\endhead
\hline
\multicolumn{4}{|c|}{continúa $\ldots$}\\
\hline
\endfoot
\hline
\endlastfoot
\textbf{Atributo}  & \textbf{Descripción} & \textbf{Importancia} & \textbf{Fuente}\\ \hline
area\_metropolitana\_rbd & Pertenencia al área metropolitana. &Media & CIAE \\ \hline
cod\_depe & Código de dependencia del establecimiento. & Alta & CIAE \\ \hline
gen\_rbd & Género del establecimiento. & Baja & CIAE \\ \hline
mat\_total & Matrícula total de alumnos. & Alta & MIME \\ \hline
prom\_alu\_cur & Promedio de alumnos por curso. & Alta & MIME \\ \hline
pago\_mat & Nivel de pago de matrícula. & Alta & MIME \\ \hline
pago\_men & Nivel de pago de mensualidad. & Alta & MIME \\ \hline
becas\_disp & Becas disponibles en el establecimiento. & Alta & MIME \\ \hline
convenio\_sep & Posee convenio de subvención escolar preferencial (SEP). & Alta & MIME \\ \hline
deportivo & Nivel deportivo del establecimiento. & Alta & MIME \\ \hline
req\_papeles & Requisitos de papeles para postular. & Media & MIME \\ \hline
req\_pruebas & Requisitos de prueba para postular. & Baja & MIME \\ \hline
req\_entrevista & Requisitos de entrevista para postular. & Alta & MIME \\ \hline
req\_pago & Requisitos de pago para postular. & Media & MIME \\ \hline
req\_otros & Requisitos de cualquier tipo que no clasifique en las categorías anteriores. & Baja & MIME \\ \hline
enf\_académico & Enfoque académico. & Media & MIME \\ \hline
enf\_valorico & Enfoque valórico. & Alto & MIME \\ \hline
enf\_laboral & Enfoque laboral. & Baja & MIME \\ \hline
enf\_otros & Enfoque de otro tipo que no clasifique en las categorías anteriores. & Baja & MIME \\ \hline
apoyo\_tutorias & Ofrece ayuda a los alumnos mediante tutorías. & Media & MIME \\ \hline
apoyo\_especialistas & Ofrece ayuda a los alumnos mediante especialistas. & Media & MIME \\ \hline
apoyo\_otros & Ofrece ayuda a los alumnos de cualquier otra forma que no clasifique en la categorías anteriores. & Baja & MIME \\ \hline
s\_basica & Establecimiento de enseñanza básica. & Alta & Derivados de datos CIAE. \\ \hline
s\_media & Establecimiento de enseñanza media. & Alta & Derivados de datos CIAE. \\ \hline
completa & Establecimiento de enseñanza completa. & Alta & Derivados de datos CIAE. \\ \hline
IDE\_rango & Índice de Desarrollo de la Educación para Todos por rango. & Alta & Calculado con datos CIAE. \\ \hline
dist\_percentil\_75 & Distancia del percentil 75 de los alumnos que asisten al establecimiento. & Alta & Calculado con datos CIAE.\\ \hline
\end{longtable} 
\end{footnotesize}

\begin{footnotesize}
\begin{longtable}{|p{0.2\textwidth}|p{0.3\textwidth}|p{0.15\textwidth}|p{0.15\textwidth}|}
\caption{Atributos seleccionados y generados para la base de datos de matrículas.}\label{tab:atributos_matriculas}\\
\hline
\endfirsthead
\caption[]{Atributos seleccionados y generados para la base de datos de matrículas. (continuación)}\\
\hline
\endhead
\hline
\multicolumn{4}{|c|}{continúa $\ldots$}\\
\hline
\endfoot
\hline
\endlastfoot
\textbf{Atributo}  & \textbf{Descripción} & \textbf{Importancia} & \textbf{Fuente} \\ \hline
area\_metropolitana\_alu & Pertenencia al área metropolitana. & Alta & CIAE \\ \hline
gen\_alu & Género del establecimiento. & Alta & CIAE \\ \hline
cod\_sec & Código del sector económico. & Baja & CIAE \\ \hline
cod\_espe & Código de especialidad. & Baja & CIAE \\ \hline
cod\_rama & Código de rama. & Baja & CIAE \\ \hline
grado\_sep & Corresponde a un nivel SEP. & Alta & CIAE \\ \hline
beneficiario\_sep & Indicador del alumno beneficiario de la SEP. & Alta & CIAE \\ \hline
criterio\_sep & Criterio por el cual se considera prioritario. & Alta & CIAE \\ \hline
sobre\_edad & Diferencia entra la edad actual y la esperada para el nivel. & Alta & Calculado con datos CIAE.\\ \hline
dist\_actual & Distancia del alumno a su establecimiento actual. & Alta & CIAE \\ \hline
pago\_mat & Nivel de pago de matrícula. & Alta & MIME \\ \hline
pago\_men & Nivel de pago de mensualidad. & Alta & MIME \\ \hline
\end{longtable} 
\end{footnotesize}

Luego de esto se transforman las variables categóricas o nominales a numéricas para poder utilizarlas en el algoritmo de clusterización junto al resto de las variables numéricas.

Se imputaron los datos faltantes dentro de cada uno de los atributos previamente escogidos mediante el algoritmo MICE \textit{(Multiple imputation by chained equations}) para eliminar todos los valores nulos.
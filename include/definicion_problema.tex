\chapter{Definici\'on del Problema}

Uno de los principales problemas que surgen al momento de querer generar políticas públicas, es determinar el público objetivo sobre el cual estas se aplicarán, es decir, en que segmento de establecimientos o estudiantes se enfocarán dichas políticas.

En la actualidad, como se señaló anteriormente, los colegio en Chile se clasifican principalmente por la dependencia administrativa, dejando fuera muchos criterios que podrían ser de gran interés. De manera similar, los alumnos que a ellos asisten no cuentan con una clasificación que describa de manera relevante los grupos y solo se dividen por el tipo de establecimiento al cual asisten o por el nivel de ingreso familiar.

Por esto es necesario identificar los diferentes perfiles de establecimientos y estudiantes presentes en Chile. Para esto se trabajó en conjunto con el Centro de Investigación Avanzada en Educación de la Universidad de Chile, el cual tiene dentro de su misión el dar soporte científico a la discusión y diseño de políticas públicas en el sector educación, para que las políticas nacionales, la gestión educativa local y la docencia en el aula estén basadas en la evidencia que genera la investigación.

El problema al cual se enfrenta el Estado al momento de realizar políticas públicas es conocer a cabalidad las características del conjunto o grupo a las cuales estas van dirigidas, lo cual es difícil de realizar con una clasificación que se basa en un solo atributo. Para realizar esta tarea sería de mayor utilidad contar con perfiles, tanto de establecimientos como de estudiantes, los cuales mediante diferentes variables puedan definir de mejor manera un grupo determinado.

El objetivo principal es establecer las relaciones existentes entre perfiles de alumnos y grupos de establecimientos para que a partir de esto se puedan generar políticas públicas de acuerdo a la realidad escolar que se vive en Chile. Además, contar con perfiles establecidos directamente de los datos facilita realizar análisis en diferentes estudios complementarios.

Para llevar a cabo esto se utilizará un algoritmo de clusterización sobre establecimientos y matrículas, que en una primera instancia permitirá determinar los atributos más importantes. En segunda instancia se incorporan variables que los relacionan y así ver si estas tienen un impacto positivo o negativo en los grupos generados anteriormente.

La realización de esto cobra gran importancia, debido a que una mala caracterización del público objetivo impide que las políticas públicas tenga los efectos esperados. Además, el no tener un buen perfil de los grupos significa no conocer como estos están formados, lo que dificulta la creación de políticas que tenga un impacto positivo. Es decir, el problema afecta a la creación de las políticas y a las efectos que estas causan.

A la fecha no existen otros estudios que realicen un trabajo similar al planteado en el presente documento, ya que ha diferencia de este buscan establecer a que establecimiento debe asistir cada alumno según diferentes funciones de utilidad.
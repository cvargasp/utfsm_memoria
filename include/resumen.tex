\chapter*{Resumen}
\addcontentsline{toc}{chapter}{Resumen}

El análisis de datos es un proceso fundamental para obtener información y con esta poder tomar diversas decisiones. En el ámbito escolar los establecimiento y estudiantes se agrupan por su dependencia o por la dependencia del colegio al que asisten. Este estudio busca encontrar mediante un algoritmo de clasificación no supervisado los diferentes grupos de colegios y alumnos, determinando cuales son sus principales características. Se comparan los resultados de un algoritmo al ser ejecutado con tres grupos de variables distintas y luego al mejor resultad se le agregan variables de relación establecimiento - matrículas y se compara con la primera versión. Los resultados más importantes son que los establecimientos y matrículas encuentran un óptimo de 4 grupos de clasificación con el grupo de menor cantidad de variables, donde una de las más relevantes a la hora de separar los grupos es el nivel de copago que existe. 

\textbf{Palabras Clave:} Análisis de datos, Aprendizaje No Supervisado, X-Means.
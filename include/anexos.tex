\renewcommand{\appendixname}{Anexos}
\renewcommand{\appendixtocname}{Anexos}
\renewcommand{\appendixpagename}{Anexos}

\appendix
\chapter{Anexo I}

\begin{longtable}{|p{5cm}|p{9cm}|}
\caption{Atributos seleccionados y generados para la base de datos de establecimientos.}\label{tab:atributos_establecimientos}\\
\hline
\endfirsthead
\caption[]{Atributos seleccionados y generados para la base de datos de establecimientos. (continuación)}\\
\hline
\endhead
\hline
\multicolumn{2}{|c|}{continúa $\ldots$}\\
\hline
\endfoot
\hline
\endlastfoot
\textbf{Atributo}  & \textbf{Descripción} \\ \hline
area\_metropolitana\_rbd & Pertenencia al área metropolitana. \\ \hline
cod\_depe & Código de dependencia del establecimiento. \\ \hline
gen\_rbd & Género del establecimiento. \\ \hline
mat\_total & Matrícula total de alumnos. \\ \hline
prom\_alu\_cur & Promedio de alumnos por curso. \\ \hline
pago\_mat & Nivel de pago de matrícula. \\ \hline
pago\_men & Nivel de pago de mensualidad. \\ \hline
becas\_disp & Becas disponibles en el establecimiento. \\ \hline
convenio\_sep & Posee convenio de subvención escolar preferencial (SEP). \\ \hline
deportivo & Nivel deportivo del establecimiento. \\ \hline
req\_papeles & Requisitos de papeles para postular. \\ \hline
req\_pruebas & Requisitos de prueba para postular. \\ \hline
req\_entrevista & Requisitos de entrevista para postular. \\ \hline
req\_pago & Requisitos de pago para postular. \\ \hline
req\_otros & Requisitos de cualquier tipo que no clasifique en las categorías anteriores. \\ \hline
enf\_académico & Enfoque académico. \\ \hline
enf\_valorico & Enfoque valórico. \\ \hline
enf\_laboral & Enfoque laboral. \\ \hline
enf\_otros & Enfoque de otro tipo que no clasifique en las categorías anteriores. \\ \hline
apoyo\_tutorias & Ofrece ayuda a los alumnos mediante tutorías. \\ \hline
apoyo\_especialistas & Ofrece ayuda a los alumnos mediante especialistas. \\ \hline
apoyo\_otros & Ofrece ayuda a los alumnos de cualquier otra forma que no clasifique en la categorías anteriores. \\ \hline
s\_basica & Establecimiento de enseñanza básica. \\ \hline
s\_media & Establecimiento de enseñanza media. \\ \hline
completa & Establecimiento de enseñanza completa. \\ \hline
IDE\_rango & Índice de desarrollo de la educación para todos por rango. \\ \hline
dist\_percentil\_75 & Distancia del percentil 75 de los alumnos que asisten al establecimiento. \\ \hline
\end{longtable} 

\chapter{Anexo II}

\begin{longtable}{|p{5cm}|p{9cm}|}
\caption{Atributos seleccionados y generados para la base de datos de matrículas.}\label{tab:atributos_matriculas}\\
\hline
\endfirsthead
\caption[]{Atributos seleccionados y generados para la base de datos de matrículas. (continuación)}\\
\hline
\endhead
\hline
\multicolumn{2}{|c|}{continúa $\ldots$}\\
\hline
\endfoot
\hline
\endlastfoot
\textbf{Atributo}  & \textbf{Descripción} \\ \hline
area\_metropolitana\_alu & Pertenencia al área metropolitana.\\ \hline
gen\_alu & Género del establecimiento. \\ \hline
area\_metropolitana\_alu & \\ \hline
cod\_sec & Código del sector económico.\\ \hline
cod\_espe & Código de especialidad. \\ \hline
cod\_rama & Código de rama. \\ \hline
grado\_sep & Corresponde a un nivel SEP.\\ \hline
beneficiario\_sep & Indicador del alumno beneficiario de la SEP.\\ \hline
criterio\_sep & Criterio por el cual se considera prioritario. \\ \hline
sobre\_edad & Diferencia entra la edad actual y la esperada para el nivel. \\ \hline
dist\_actual & Sitancia del alumno a su establecimiento actual\\ \hline
pago\_mat & Nivel de pago de matrícula. \\ \hline
pago\_men & Nivel de pago de mensualidad. \\ \hline
\end{longtable} 

\begin{table}[]
\centering
\caption{Resumen }
\label{my-label}
\begin{tabular}{|l|l|l|l|l|}
\hline
Año  & CIAE & MIME & \% de coincidencia   \\ \hline
2013 & 2110 & 2040 & 96,68 \\ \hline
2014 & 2095 & 2049 & 97,8  \\ \hline
2015 & 2088 & 2057 & 98,52 \\ \hline
2016 & 2068 & 2061 & 99,66 \\ \hline
\end{tabular}
\end{table}

